\documentclass[a4paper]{article}

\usepackage[l2tabu, orthodox]{nag}

\usepackage[utf8]{inputenc}
\usepackage[T1]{fontenc}

\usepackage[ngerman]{babel}

\usepackage{amsmath}
\usepackage{amssymb}
%\usepackage{amsthm}
\usepackage{mathtools}
\usepackage{physics}
\usepackage{dsfont}

\usepackage[framed]{ntheorem}

\usepackage{csquotes}
\usepackage{lmodern}
\usepackage{microtype}
\usepackage{enumitem}

\usepackage{faktor}

\usepackage{parskip}
\usepackage{multicol}

\usepackage{array}
\usepackage{blindtext}
\usepackage{float}

\usepackage[hidelinks]{hyperref}

\usepackage[left=1.8cm, right=1.8cm, top=1.8cm, bottom=2.5cm]{geometry}

\newcounter{Sec}

\theoremstyle{marginbreak}
\theorembodyfont{\normalfont}
\newtheorem{definition}{Definition}[Sec]
\newtheorem{satz}[definition]{Satz}
\newtheorem{defsatz}[definition]{Definition und Satz}
\newtheorem{verfahren}[definition]{Verfahren}
\newtheorem{defver}[definition]{Definition und Verfahren}
\newtheorem{defsatzver}[definition]{Definition, Satz und Verfahren}
\newtheorem{satzver}[definition]{Satz und Verfahren}
\newtheorem{folgerung}[definition]{Folgerung}

\MakeOuterQuote{"}
\DeclareMathOperator{\ffa}{ffa}

\newcommand{\sep}{%
	\rule{\textwidth}{0.3pt}%
	\stepcounter{Sec}%
}
\newcommand{\defiff}{\mathrel{\vcentcolon\Longleftrightarrow}}

\newcommand{\en}{~(n\to\infty)}
\newcommand{\series}[1][1]{\sum_{n=#1}^\infty}
\newcommand{\ps}[1][a]{\series[0]#1_n(x-x_0)^n}
\renewcommand{\d}{\dd}
\renewcommand{\P}{\mathcal{P}}
\newcommand{\R}{\mathbb{R}}
\newcommand{\C}{\mathbb{C}}
\newcommand{\A}{\mathfrak{A}}
\newcommand{\B}{\mathfrak{B}}
\renewcommand{\i}{\mathrm{i}}
\newcommand{\D}{\mathbb{D}}
\newcommand{\compl}[1]{#1^\mathsf{c}}
\newcommand{\sa}{$\sigma$-Algebra}
\newcommand{\Ri}{\mathfrak{R}}

\renewcommand{\L}[1]{\mathfrak{L}^{#1}(X)}
\newcommand{\LL}[1]{L^{#1}(X)}

\newcolumntype{M}[1]{>{\centering\arraybackslash}m{#1}}
\newcolumntype{N}{@{}m{0pt}@{}}

\setlength\columnsep{1.5cm}

\DeclareMathOperator{\Spek}{Spek}
\DeclareMathOperator{\Kern}{Kern}
\DeclareMathOperator{\arsinh}{arsinh}
\DeclareMathOperator{\arcosh}{arcosh}
\DeclareMathOperator{\artanh}{artanh}
\DeclareMathOperator{\ddiv}{div}
\DeclareMathOperator{\rot}{rot}
\DeclareMathOperator*{\esssup}{ess\,sup}
\DeclareMathOperator{\supp}{supp}
\DeclareMathOperator{\Arg}{Arg}
\DeclareMathOperator{\Log}{Log}
\DeclareMathOperator{\dist}{dist}
\DeclareMathOperator{\spur}{spur}

\DeclarePairedDelimiterX\set[1]\lbrace\rbrace{\def\given{\;\delimsize\vert\;}#1}

\newcommand\restr[2]{{#1_{\mkern 1mu \vrule height 2ex\mkern2mu #2}}}

\author{Philipp Schaback}
\title{Numerik - Lösungsverfahren}
\begin{document}
	\textsc{LR-Zerlegung ohne Pivot}
	\sep
	\begin{satz}[Existenz]
		Eine Matrix $A \in \mathbb{R}^{\mathbb{N}x\mathbb{N}} $ besitzt genau dann ein LR-Zerlegung von A, wenn alle Hauptuntermatrizen $A[1:n,1:n]$ regulär sind.
	\end{satz}
	\begin{satz}[Hinreichende Bedingung 1]
		Ist eine Matrix $A \in \mathbb{R}^{\mathbb{N}x\mathbb{N}} $ strikt diagonal-dominant, d.h. es gilt:
		\begin{description}
			\item $ |A[n,n]| > \sum_{k=1 k\ne n}^{N} |A[n,k]|$ für $n=1,...,N$
		\end{description}
		dann existiert eine LR-Zerlegung für A.
	\end{satz}
	\begin{satz}[Hinreichende Bedingung 2]
		Ist eine Matrix $A \in \mathbb{R}^{\mathbb{N}x\mathbb{N}} $ positiv definit, d.h. es gilt:
		\begin{description}
			\item $ x^TAx > 0 $ für alle $ x \in \mathbb{R}^\mathbb{N},x \ne 0$
		\end{description}
		dann existiert eine LR-Zerlegung für A.
	\end{satz}
	\begin{verfahren}
		Zerlege $A$ in $A=LR$ mit $ L \in \mathbb{R}^{\mathbb{N}x\mathbb{N}}$ normierte, untere Dreiecksmatrix und $ R \in \mathbb{R}^{\mathbb{N}x\mathbb{N}}$ reguläre obere Dreiecksmatrix, d.h.:
		\begin{description}
			\item $ L =
			\begin{pmatrix}
				1 & 0 & 0 & 0 \\
				* & 1 & 0 & 0 \\
				* & * & 1 & 0 \\
				* & * & * & 1
			\end{pmatrix}
			$\space
			$
			R = 
			\begin{pmatrix}
			* & * & * & * \\
			0 & * & * & * \\
			0 & 0 & * & * \\
			0 & 0 & 0 & *
			\end{pmatrix}
			$ mit $det(R) \ne 0$
		\end{description}
		Berechne dann mit Vorwärts-Substitution $ Ly = b$ und anschließend $ Rx = y$ mit Rückwärts-Substitution.
	\end{verfahren}
	\textsc{Cholesky-Zerlegung}
	\sep
	
\end{document}